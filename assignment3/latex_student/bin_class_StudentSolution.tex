% !TEX root = HW3.tex

\newcommand{\binStudSolA}{
%%%%%%%%%%%%%%%%%%%%%%%%%%%%%%%%%%%%
%%
%%.   YOUR SOLUTION FOR PROBLEM A BELOW THIS COMMENT
%%
%%%%%%%%%%%%%%%%%%%%%%%%%%%%%%%%%%%%
$\mathbf{y} = sign(\mathbf{w}^T\mathbf{x})$. Positive values will be treated as 1 and negative\\values will be treated as -1.
\vspace{3cm}
}

\newcommand{\binStudSolB}{
%%%%%%%%%%%%%%%%%%%%%%%%%%%%%%%%%%%%
%%
%%.   YOUR SOLUTION FOR PROBLEM B BELOW THIS COMMENT
%%
%%%%%%%%%%%%%%%%%%%%%%%%%%%%%%%%%%%%
$\mathbf{y} = sign(g(\mathbf{w}^T\mathbf{x})-0.5)$. Values above 0.5 will be treated as 1 and\\values under 0.5 will be treated as -1.
\vspace{3cm}
}

\newcommand{\binStudSolC}{
%%%%%%%%%%%%%%%%%%%%%%%%%%%%%%%%%%%%
%%
%%.   YOUR SOLUTION FOR PROBLEM C BELOW THIS COMMENT
%%
%%%%%%%%%%%%%%%%%%%%%%%%%%%%%%%%%%%%
\begin{align*}
g(a)&=\frac{1}{1+e^{-a}}\\
\frac{\partial g(a)}{\partial a}&=\frac{\partial }{\partial a}\frac{1}{1+e^{-a}}\\
\frac{\partial g(a)}{\partial a}&=\frac{e^{-a}}{(1+e^{-a})^2}\\
\frac{\partial g(a)}{\partial a}&=\frac{1}{1+e^{-a}}\frac{e^{-a}}{1+e^{-a}}\\
\frac{\partial g(a)}{\partial a}&=g(a)\frac{1+e^{-a}-1}{1+e^{-a}}\\
\frac{\partial g(a)}{\partial a}&=g(a)(\frac{1+e^{-a}}{1+e^{-a}}-\frac{1}{1+e^{-a}})\\
\frac{\partial g(a)}{\partial a}&=g(a)(1-g(a))\\
\end{align*}
}

\newcommand{\binStudSolD}{
%%%%%%%%%%%%%%%%%%%%%%%%%%%%%%%%%%%%
%%
%%.   YOUR SOLUTION FOR PROBLEM D BELOW THIS COMMENT
%%
%%%%%%%%%%%%%%%%%%%%%%%%%%%%%%%%%%%%
\begin{align*}
k^{(i)} :=& g(\mathbf{w}^T\mathbf{x}^{(i)})\\
\nabla_\mathbf{w}=&-(\mathbf{y}^{(i)}-k^{(i)})k^{(i)}(1-k^{(i)})\mathbf{x}^{(i)}\\
\mathbf{w}_{n+1}=&\mathbf{w}_{n} + \eta\nabla_{\mathbf{w}_n}
\end{align*}
}

\newcommand{\binStudSolE}{
%%%%%%%%%%%%%%%%%%%%%%%%%%%%%%%%%%%%
%%
%%.   YOUR SOLUTION FOR PROBLEM E BELOW THIS COMMENT
%%
%%%%%%%%%%%%%%%%%%%%%%%%%%%%%%%%%%%%
Whenever we use logistic regression we do so under the assumption\\that all the independent variables are truly independent of one another and are\\identically distributed.
\vspace{3cm}
}

